\section{Package Design and Architecture}
The \glsname{caldav} format, short for the Calendaring Extensions to \gls{webdav} as introduced in \cite{caldav-rfc} defines three types of entities: VEVENTs, VTODOs and VJOURNALs.
These entities are organised into calendars, for our purposes these could be thought of as different todo lists.
\textit{Melon} interacts with CalDAV servers and objects through Python's \texttt{caldav} package.
A decent amount of the code in \texttt{melon} and \texttt{melongui} is concerned with the interaction from the package to these objects.
Within the scope of this report, we will focus on a smaller version of these VTODO objects, created for a swift interface to the scheduler algorithm implementations.

This small object version, containing data relevant to the scheduling mechanism, looks like this:
\begin{minted}{python}
    import dataclasses
    from datetime import datetime

    @dataclasses.dataclass
    class Task:
        uid: str  # unique identifier of the task
        duration: float  # estimated, in hours
        priority: int  # between 1 and 9
        location: int  # number indicating the location, 0 is "hybrid"
        due: datetime | None  # when the task is due
  \end{minted}

So each task has an associated UID, duration, priority, location and due date.
UIDs are useful because they make value collisions very unlikely.
Which is not to say that these should not be checked, but if two separate calendar clients that each generated a set of UIDs, connected to a server, it is very unlikely to have to resolve potential conflicts.

As mentioned above, the energy we minimise (to schedule the tasks) is a combination of four properties.
To obtain our energy function, we propose numerical expressions for each of them (again, the lower, the better the state) and then perform a weighted sum over all four.
Informally stated, the function we minimise is approximately given by
\begin{align}
  \label{eq:energy} E(\vec{x}) = \; & \mathrm{slot~end}_N - \mathrm{slot~start}_1 + \sum_{j=1}^{N} \identity_{\mathrm{slot~end}_j > \mathrm{due}_j} \cdot 100                    \\
  \nonumber                         & + \sum_{j=2}^{n} (1 - \identity_{\mathrm{location}_{j-1}, \mathrm{location}_{j}}) \cdot 30 + \sum_{j=1}^{N} j \cdot \mathrm{priority}_j\,.
\end{align}
Results of the simulation may be found in \Cref{sec:results}.

\subsection{Four Different Implementations}
In order to compare runtimes, the same algorithm was implemented four times.
Once in pure Python, once using the \texttt{numba} library and once in Rust and in C++.
Numba uses Just-In-Time compilation to speed up subsequent calls of a subroutine.
Rust and C++ are good choices for iterative procedures as they allow for low-level access to the implementation. Bindings are provided using \texttt{rust-cpython} and \texttt{pybind11}.
