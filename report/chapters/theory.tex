\section{Primer on the Underlying Theory}
\label{sec:theory}
Many complex problems cannot be solved using analytical methods due to, for instance, their discrete nature.
\gls{mcmc} methods with transition probabilities allow us to explore a huge state space regardless and minimise a function (energy $E$) therein.
There are many use cases in physics, such as for the simulation of the Ising model, where the term \textit{energy} originates from.

\begin{algorithm}[language=pseudo,caption={\centering The Metropolis-Hastings sweep() sub-routine \parencite{metropolis, hastings}},basicstyle=\footnotesize]
for $N^2$ many times, repeat
sample a candidate $\vec{x}^*$.
set $\vec{x}^{n+1} = \vec{x}^*$ with acceptance probability
  $p_{\rm accept} = \min\left(1, \e^{-\beta (E^{n+1} - E^n)}\right)\,,$ with $\beta \in \R^+$ a transition factor.
Otherwise, let $\vec{x}^{n+1} = \vec{x}^{n}$.
\end{algorithm}

Which is a subroutine to an outer iteration, a technique commonly referred to as Simulated Annealing.
The idea of the iterative procedure above is to start from an initial state $\vec{x}^0$, permute it slightly and then accept that new proposal with a probability proportional to the exponential of their energy difference.
If the proposal's energy is better (lower) than the current state's energy, the acceptance probability is $1$.
This allows us to explore state space, but not get stuck in local minima as there is always a non-zero chance for the iteration to escape the local minimum.

In our specific case, we will minimise the function $E(\vec{x})$ given in \Cref{eq:energy} based on the four properties stated above, using Metropolis-Hastings with Simulated Annealing.

\begin{algorithm}[language=pseudo,caption={\centering Simulated Annealing},basicstyle=\footnotesize]
let k = 1
until convergence, repeat
set the temperature $T = T_0 k^{q}$ and therefore $\beta = \frac{1}{T}$.
perform a sweep()
evaluate $\langle E\rangle$ and $\left\langle\Delta E^2\right\rangle$.
set k = k + 1
\end{algorithm}

The key idea is to lower the temperature over the course of the simulation to reduce the transition probability.
For each temperature $T$ we evaluate the average of the energy
$$\langle E\rangle \simeq \frac{1}{n} \sum_{\vec{x}} E(\vec{x}), \quad \text { and } \quad\left\langle E^2\right\rangle \simeq \frac{1}{n} \sum_{\vec{x}} E^2(\vec{x})$$
with $n$ the number of iterations for this temperature, hence the variance is given by
$$\left\langle\Delta E^2\right\rangle:=\left\langle E^2\right\rangle-\langle E\rangle^2 \,.$$
When the variance subceeds a certain threshold, one could stop the iteration.
